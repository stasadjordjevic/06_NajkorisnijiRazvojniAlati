\documentclass[a4paper]{article}
\usepackage{color}
\usepackage{url}
\usepackage{graphicx}
\usepackage{caption}
\usepackage[T2A]{fontenc}
\usepackage[utf8]{inputenc}
\usepackage{graphicx}
\usepackage[english,serbian]{babel}
\usepackage[unicode]{hyperref}
\hypersetup{colorlinks,citecolor=blue,filecolor=green,linkcolor=blue,urlcolor=blue} 
\title{Najkorisniji razvojni alati \\
\normalsize Seminarski rad u okviru kursa\\ Metodologija stručnog i naučnog rada
\\Matematički fakultet}
\author{Staša Đorđević \and
Lazar Savić \and
Đurđa Milošević \and
Bogdan Tomić} 
\date{18. decembar 2025.}

\begin{document}
\maketitle

\abstract{ ovde pišemo abstrakt.}
\tableofcontents

\newpage
\section{Uvod}

ovde pišemo uvod

\section{O razvojnim alatima}

\section{Metodologija}

Istraživanje je sprovedeno korišćenjem anonimne onlajn ankete. 
Metodološki pristup je dominantno kvantitativan, budući da su 
podaci prikupljeni putem strukturiranih pitanja i analizirani 
primenom deskriptivne statistike, uz ograničene kvalitativne 
uvide dobijene analizom odgovora otvorenog tipa.

\subsection {Uzorkovanje}

U istraživanju je učestvovalo ukupno 43 ispitanika koji se bave
programiranjem. Uzorak je obuhvatio studente tehničkih i srodnih
fakulteta, kao i zaposlene profesionalce u IT sektoru. Uzorkovanje je
sprovedeno metodom pogodnog (neprobabilističkog) uzorka, s 
obzirom na to da su ispitanici dobrovoljno učestvovali u anketi
i bili dostupni autorima istraživanja putem onlajn kanala.

\subsection {Struktura uzorka}

Struktura uzorka analizirana je kroz demografske i profesionalne 
karakteristike ispitanika, pri čemu su prikupljeni podaci o dužini 
iskustva u programiranju, pohađanim fakultetima, 
kao i polu. Među ispitanicima, 27 je bilo muškog, a 16 ženskog pola. 
Pregled ostalih demografskih karakteristika prikazan je na slikama 
\ref{fig:iskustvo} i \ref{fig:fakultet}.

\begin{figure}[h]
    \centering
    \includegraphics[width=0.7\textwidth]{"Slike/duzina iskustva.png"}
    \caption{Raspodela ispitanika po dužini iskustva u programiranju.}
    \label{fig:iskustvo}
\end{figure}

\begin{figure}[h]
    \centering
    \includegraphics[width=0.8\textwidth]{"Slike/fakulteti.png"}
    \caption{Broj ispitanika po fakultetima.}
    \label{fig:fakultet}
\end{figure}

\subsection {Struktura ankete}

Anketa je sadržala 16 pitanja, od kojih je 12 bilo zatvorenog tipa, 
dok su četiri pitanja omogućavala slobodne odgovore. Pitanja su 
organizovana po celinama: 
\begin{itemize}
    \item \textbf{Demografska i profesionalna pitanja} - dužina iskustva u programiranju,
    pol, pohađani fakultet, trenutna pozicija u IT sektoru. Pitanja su bila zatvorenog tipa
    sa ponuđenim odgovorima.
    
    \item \textbf{Pitanja o korišćenju razvojnih alata} – najčešće korišćeni
    IDE i tekstualni editori, sistemi za verzionisanje, 
    preferencije između GitHub i GitLab, glavni razvojni alat, 
    razlozi izbora i problemi sa alatima. Većina pitanja bila je 
    zatvorenog tipa, dok su argumentacija izbora i opis problema bili 
    otvorenog tipa radi dodatnih komentara.

    \item \textbf{Faktori izbora razvojnog alata} - pitanja o važnosti 
    karakteristika alata, uticaju dugogodišnjih navika, ranijim promenama
     glavnog alata, razlozima za prelazak na novi alat i faktorima koji 
     bi motivisali tu promenu. Za većinu ovih pitanja korišćena je 
     \textit{Likert skala} od 1 do 5.
\end{itemize}

Svi odgovori su bili anonimni, a učestvovanje u anketi dobrovoljno.

\subsection {Analiza dobijenih rezultata}
% koje tehnike smo koristili za analizu prikupljenih podataka


\section{Rezultati}

\section{Diskusija}


\section{Zaključak}
ovde pišemo zaključak


\addcontentsline{toc}{section}{Literatura}
\renewcommand{\refname}{Literatura}
\begin{thebibliography}{10}
\bibliographystyle{unsrt}
\bibitem{naziv stavke za referisanje} Naziv izvora  at:\\ \url{link} 
\end{thebibliography}
\end{document}