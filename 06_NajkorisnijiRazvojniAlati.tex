\documentclass[a4paper]{article}
\usepackage{color}
\usepackage{url}
\usepackage{graphicx}
\usepackage{caption}
\usepackage[T2A]{fontenc}
\usepackage[utf8]{inputenc}
\usepackage{graphicx}
\usepackage[english,serbian]{babel}
\usepackage[unicode]{hyperref}
\hypersetup{colorlinks,citecolor=blue,filecolor=green,linkcolor=blue,urlcolor=blue}
\usepackage{float} 
\usepackage{placeins}
\title{Najkorisniji razvojni alati \\
\normalsize Seminarski rad u okviru kursa\\ Metodologija stručnog i naučnog rada
\\Matematički fakultet}
\author{Staša Đorđević \and
Lazar Savić \and
Đurđa Milošević \and
Bogdan Tomić} 
\date{18. decembar 2025.}

\begin{document}
\maketitle

\abstract{ ovde pišemo abstrakt.}
\tableofcontents

\newpage
\section{Uvod}

Razvojni alati predstavljaju jedan od najznačajnijih faktora koji utiču na produktivnost i efikasnost programera i, kao takvi, čine neizostavni deo njihovog svakodnevnog rada. Imajući u vidu veliki broj dostupnih razvojnih alata, koji se međusobno razlikuju po funkcionalnostima, performansama, mogućnostima prilagođavanja i korisničkom iskustvu, izbor najkorisnijeg alata nije jednoznačan. On često zavisi od ličnih preferencija korisnika, prethodnog iskustva, kao i specifičnih zahteva projekta ili okruženja u kojem se alat koristi.

Predmet ovog rada jeste analiza upotrebe razvojnih alata, sa fokusom na integrisana razvojna okruženja (IDE), tekstualne editore koda i sisteme za verzionisanje, kao tri osnovne kategorije alata koje se najčešće koriste u procesu razvoja softvera. Cilj rada jeste da se ispita koje konkretne alate iz navedenih kategorija studenti i iskusniji programeri najčešće koriste, koje faktore smatraju najvažnijim prilikom njihovog izbora, kao i šta ih motiviše da pređu na novi alat ili ostanu pri postojećem. Pored toga, poseban akcenat stavljen je na ulogu navike u korišćenju određenog alata, kao i na zadovoljstvo korisnika alatima koje trenutno koriste, u cilju identifikovanja potencijalnih unapređenja samih alata.

Istraživanje je sprovedeno putem onlajn ankete, pri čemu su prikupljeni kvantitativni i kvalitativni podaci o navikama, stavovima i iskustvima ispitanika u vezi sa pomenutim razvojnim alatima. Značaj ovog istraživanja ogleda se u pružanju sistematičnog uvida u način na koji korisnici percipiraju i biraju razvojne alate, kao i u boljem razumevanju faktora koji utiču na donošenje odluka u procesu izbora alata u realnom akademskom i profesionalnom kontekstu.

\section{O razvojnim alatima}

Razvoj softvera podrazumeva upotrebu različitih alata koji programerima omogućavaju efikasnije pisanje, organizaciju, testiranje i održavanje programskog koda. U ovom radu razmatraju se tri osnovne kategorije razvojnih alata: integrisana razvojna okruženja (IDE), tekstualni editori koda i sistemi za verzionisanje.

\subsection{Integrisana razvojna okruženja (IDE)}

Integrisana razvojna okruženja (IDE -- \textit{Integrated Development Environment}) predstavljaju softverske alate koji objedinuju više funkcionalnosti potrebnih za razvoj softvera u jedinstveno radno okruženje. Tipične komponente IDE-a uključuju editor koda, kompajler ili interpreter, alate za dinamičku analizu koda, kao i dodatne funkcionalnosti za upravljanje projektima i testiranje.

Glavna prednost integrisanih razvojnih okruženja ogleda se u visokom stepenu automatizacije i podrške programeru. Funkcionalnosti kao što su automatsko dopunjavanje koda, statička analiza, upozorenja na potencijalne greške i integrisana dokumentacija mogu značajno ubrzati proces razvoja i smanjiti broj grešaka u kodu.

Primeri integrisanih razvojnih okruženja koji se često koriste u praksi uključuju Visual Studio, IntelliJ IDEA, PyCharm i Android Studio.

Sa druge strane, integrisana razvojna okruženja mogu zahtevati veće hardverske resurse i biti složenija za prilagođavanje individualnim potrebama korisnika. Uprkos tome, njihova sveobuhvatnost čini ih posebno pogodnim kako za profesionalno okruženje, tako i za obrazovne svrhe.

\subsection{Tekstualni editori koda}

Tekstualni editori koda predstavljaju alate namenjene pisanju i uređivanju izvornog koda, sa fokusom na jednostavnost i fleksibilnost. Za razliku od integrisanih razvojnih okruženja, tekstualni editori obično nude osnovne funkcionalnosti kao što su isticanje sintakse, jednostavna navigacija kroz fajlove i minimalna podrška za analizu koda.

Jedna od ključnih prednosti tekstualnih editora jeste njihova brzina rada i mala potrošnja sistemskih resursa. Mnogi editori omogućavaju proširivanje funkcionalnosti putem dodataka ili ekstenzija, čime se mogu prilagoditi različitim programskim jezicima i stilovima rada. Ovakav modularni pristup korisnicima pruža veću kontrolu nad razvojnim okruženjem.

Primeri često korišćenih tekstualnih editora koda su Sublime Text, Kate i Vim.

\subsection{Sistemi za verzionisanje}

Sistem za verzionisanje je sistem koji je zadužen za čuvanje i kontrolisanje izmena programskog koda, odnosno za upravljanje različitim verzijama softvera koji se razvija. Njihova osnovna uloga jeste čuvanje istorije izmena, omogućavanje povratka na prethodne verzije koda i olakšavanje saradnje više korisnika na istom projektu. 

Upotrebom sistema za verzionisanje, programeri mogu paralelno raditi na različitim delovima projekta, uz kasnije objedinjavanje izmena. Ovi sistemi značajno smanjuju rizik od gubitka podataka i doprinose boljoj timskoj organizaciji.

Sistemi za verzionisanje mogu se podeliti na centralizovane i distribuirane, u zavisnosti od načina na koji se čuva i deli istorija izmena. Centralizovani sistemi za verzionisanje podrazumevaju postojanje centralnog repozitorijuma na kojem se čuva kompletna istorija projekta, dok korisnici pristupaju tom repozitorijumu radi preuzimanja i slanja izmena. Ovakav pristup olakšava centralnu kontrolu, ali može predstavljati problem u slučaju nedostupnosti centralnog servera.

Nasuprot tome, distribuirani sistemi za verzionisanje omogućavaju svakom korisniku da poseduje potpunu lokalnu kopiju repozitorijuma. Ovakav model omogućava rad bez stalne konekcije sa centralnim serverom, veću fleksibilnost i pouzdanost, kao i efikasniji paralelni razvoj. Zbog ovih karakteristika, distribuirani sistemi za verzionisanje danas su dominantni u savremenom razvoju softvera.

Najpoznatiji primer distribuiranog sistema za verzionisanje je Git.


\section{Metodologija}

Istraživanje je sprovedeno korišćenjem anonimne onlajn ankete. 
Metodološki pristup je dominantno kvantitativan, budući da su 
podaci prikupljeni putem strukturiranih pitanja i analizirani 
primenom deskriptivne statistike, uz ograničene kvalitativne 
uvide dobijene analizom odgovora otvorenog tipa.

\subsection {Uzorkovanje}

U istraživanju je učestvovalo ukupno 43 ispitanika koji se bave
programiranjem. Uzorak je obuhvatio studente tehničkih i srodnih
fakulteta, kao i zaposlene profesionalce u IT sektoru. Uzorkovanje je
sprovedeno metodom pogodnog (neprobabilističkog) uzorka, s 
obzirom na to da su ispitanici dobrovoljno učestvovali u anketi
i bili dostupni autorima istraživanja putem onlajn kanala.

\subsection {Struktura uzorka}

Struktura uzorka analizirana je kroz demografske i profesionalne 
karakteristike ispitanika, pri čemu su prikupljeni podaci o dužini 
iskustva u programiranju, pohađanim fakultetima, 
kao i polu. Među ispitanicima, 27 je bilo muškog, a 16 ženskog pola. 
Pregled ostalih demografskih karakteristika prikazan je na slikama 
\ref{fig:iskustvo} i \ref{fig:fakultet}.

\begin{figure}[h]
    \centering
    \includegraphics[width=0.7\textwidth]{"Slike/duzina iskustva.png"}
    \caption{Raspodela ispitanika po dužini iskustva u programiranju.}
    \label{fig:iskustvo}
\end{figure}

\begin{figure}[h]
    \centering
    \includegraphics[width=0.8\textwidth]{"Slike/fakulteti.png"}
    \caption{Broj ispitanika po fakultetima.}
    \label{fig:fakultet}
\end{figure}

\subsection {Struktura ankete}

Anketa je sadržala 16 pitanja, od kojih je 12 bilo zatvorenog tipa, 
dok su četiri pitanja omogućavala slobodne odgovore. Pitanja su 
organizovana po celinama: 
\begin{itemize}
    \item \textbf{Demografska i profesionalna pitanja} - dužina iskustva u programiranju,
    pol, pohađani fakultet, trenutna pozicija u IT sektoru. Pitanja su bila zatvorenog tipa
    sa ponuđenim odgovorima.
    
    \item \textbf{Pitanja o korišćenju razvojnih alata} – najčešće korišćeni
    IDE i tekstualni editori, sistemi za verzionisanje, 
    preferencije između GitHub i GitLab, glavni razvojni alat, 
    razlozi izbora i problemi sa alatima. Većina pitanja bila je 
    zatvorenog tipa, dok su argumentacija izbora i opis problema bili 
    otvorenog tipa radi dodatnih komentara.

    \item \textbf{Faktori izbora razvojnog alata} - pitanja o važnosti 
    karakteristika alata, uticaju dugogodišnjih navika, ranijim promenama
     glavnog alata, razlozima za prelazak na novi alat i faktorima koji 
     bi motivisali tu promenu. Za većinu ovih pitanja korišćena je 
     \textit{Likert skala} od 1 do 5.
\end{itemize}

Svi odgovori su bili anonimni, a učestvovanje u anketi dobrovoljno.

\subsection {Analiza dobijenih rezultata}
% koje tehnike smo koristili za analizu prikupljenih podataka


\section{Rezultati}


U ovom poglavlju izloženi su podaci prikupljeni sprovedenom anketom, koji daju uvid u najkorisnije razvojne alate. Uzorak obuhvata ispitanike različitih nivoa stručnosti od studenata do zaposlenih u IT industriji što omogućava analizu tržišta iz više perspektiva.

Prikazani rezultati su podeljeni u nekoliko logičkih celina:
\begin{itemize}
    \item Ocene važnosti ključnih funkcionalnosti (brzina, UX, AI asistenti) u zavisnosti od kategorije alata.
    \item Statistički podaci o korišćenju razvojnih okruženja (IDE) s obzirom na pol i radni status ispitanika.
    \item Odnos između godina radnog iskustva i spremnosti na promenu primarnog alata.
    \item Specifičnosti u zahtevima studenata sa MATF-a, FON-a i RAF-a.
\end{itemize}

Svi grafikoni su generisani na osnovu odgovora ispitanika, a prateća analiza se fokusira na uočene trendove i korelacije u dobijenim podacima.

\subsection{Analiza faktora važnosti pri izboru alata}

U ovom delu rada fokusiramo se na kriterijume koji programeri smatraju presudnim prilikom odabira radnog okruženja. Na slici \ref{fig:bitni_faktori} prikazani su rezultati za 11 različitih kategorija alata.

\begin{figure}[H] % [H] fiksira sliku tačno ovde
    \centering
    \includegraphics[width=1\textwidth]{Slike/koji_faktori_su_najbitniji_kirosnicima_razlicitih_alata.png}
    \caption{Važnost faktora prilikom izbora razvojnog alata}
    \label{fig:bitni_faktori}
\end{figure}

Za svaki od 11 analiziranih razvojnih alata izračunata je prosečna ocena kvaliteta $\bar{X}_{A}$, koja predstavlja aritmetičku sredinu šest ključnih kriterijuma (brzina, jednostavnost, univerzalnost, dostupnost dodataka, vizuelni izgled i AI trendovi). Formula po kojoj je vršeno računanje za svaki alat $A$ je:

\begin{equation}
\bar{X}_{A} = \frac{f_{brzina} + f_{jednost} + f_{univ} + f_{plugin} + f_{visual} + f_{AI}}{6}
\end{equation}

Na osnovu ove metrike, možemo preciznije rangirati alate prema ukupnom zadovoljstvu korisnika:

\begin{itemize}
    \item \textbf{Cursor:} $\bar{X} \approx 4.41$ — Najviše rangiran alat, prvenstveno zbog maksimalnih ocena u kategorijama brzine i praćenja AI trendova.
    \item \textbf{CLion i IntelliJ:} $\bar{X} \approx 3.85$ — Ovi alati zadržavaju visok prosek zbog ujednačenih ocena u svim kategorijama, uprkos nižem rezultatu za AI trendove.
    \item \textbf{VS Code:} $\bar{X} \approx 3.78$ — Balansiran alat sa visokim ocenama za dostupnost dodataka i brzinu.
    \item \textbf{PyCharm:} $\bar{X} \approx 4.00$ — Visok prosek zahvaljujući ujednačenosti svih faktora, gde nijedna ocena ne pada ispod 3.
    \item \textbf{NeoVim i Vim:} $\bar{X} \approx 3.42$ — Specifični alati koji imaju ekstremno visoke ocene za brzinu, ali niži prosek zbog vizuelnog izgleda i AI integracija.
    \item \textbf{Visual Studio:} $\bar{X} \approx 3.41$ — Solidan prosek koji održava univerzalnost i brzina, dok su AI trendovi slabije ocenjeni.
\end{itemize}

Ovakvo računanje pokazuje da moderni alati poput \textit{Cursor-a} trenutno nude najkompletniji paket funkcionalnosti, dok tradicionalni editori (Vim, NeoVim) ostaju specijalizovani alati sa visokim performansama, ali užim fokusom na moderne vizuelne i AI standarde.


\subsection{Analiza korišćenja razvojnih okruženja u odnosu na pol}

Ova podsekcija analizira distribuciju najpopularnijih razvojnih alata (IDE) među muškim i ženskim ispitanicima. Podaci su prikupljeni kako bi se utvrdilo da li postoji značajna korelacija između pola i izbora softvera za programiranje.

\begin{figure}[H]
    \centering
    \includegraphics[width=0.95\textwidth]{Slike/koriscenje_razvojnih_alata_ide_prema_polu.png}
    \caption{Korišćenje IDE alata u zavisnosti od pola ispitanika}
    \label{fig:ide_pol}
\end{figure}

Na osnovu podataka prikazanih na slici \ref{fig:ide_pol}, uočavamo odnos  razvojnih alata u odnosu na pol ispitanika:

\begin{itemize}
    \item \textbf{Dominantni alati:} \textit{Visual Studio} i \textit{VS Code} su najzastupljeniji alati u uzorku. Interesantno je da oba rešenja imaju identičnu strukturu korisnika, po 21 muški i 15 ženskih ispitanika, što ukazuje na njihovu široku prihvaćenost bez obzira na pol.
    \item \textbf{Uravnotežena zastupljenost:} Alat \textit{IntelliJ IDEA} izdvaja se po potpunoj ravnopravnosti, sa tačno 10 muških i 10 ženskih korisnika, što ga čini statistički najneutralnijim alatom u pogledu preferencija polova.
    \item \textbf{Specifične razlike:} 
    \begin{itemize}
        \item Alati poput \textit{CLion}-a i \textit{PyCharm}-a pokazuju veću zastupljenost kod ženskih ispitanika u odnosu na njihov ukupan broj korisnika.
        \item S druge strane, rešenja kao što su \textit{Cursor} i \textit{Eclipse} u ovom istraživanju su koristili isključivo muški ispitanici.
    \end{itemize}
\end{itemize}

Ovi rezultati sugerišu da su industrijski standardi, poput Microsoft-ove porodice alata, univerzalno prihvaćeni u celoj zajednici. Razlike u preferencijama postaju uočljivije kod alata koji su usko specijalizovani za određene programske jezike ili kod novijih rešenja baziranih na veštačkoj inteligenciji.


\subsection{Analiza preferencija Git platformi kod korisnika IntelliJ IDEA okruženja}

U okviru istraživanja, specifičan fokus je stavljen na korisnike koji kao svoje primarno ili često razvojno okruženje koriste \textit{IntelliJ IDEA}. Cilj je bio utvrditi da li izbor profesionalnog IDE alata korelira sa izborom platforme za upravljanje verzijama koda (Git). Rezultati su prikazani na slici \ref{fig:git_intellij}.

\begin{figure}[H]
    \centering
    \includegraphics[width=0.75\textwidth]{Slike/github_vs_gitlab_korisnici_intellij.png}
    \caption{Odnos GitHub i GitLab korisnika među korisnicima IntelliJ IDEA okruženja}
    \label{fig:git_intellij}
\end{figure}

Na osnovu vizuelnog prikaza sa slike \ref{fig:git_intellij}, uočavaju se jasni trendovi u izboru platformi za kontrolu verzija među korisnicima \textit{IntelliJ IDEA} okruženja:

\begin{itemize}
    \item \textbf{Dominacija GitHub platforme:} Velika većina ispitanika, tačnije 80\%, opredelila se za \textit{GitHub}. Ovako visok procenat potvrđuje da je GitHub primarni standard u zajednici, verovatno zbog njegove široke zastupljenosti u open-source projektima i industriji.
    \item \textbf{Zastupljenost GitLab-a:} Preostalih 20\% korisnika koristi \textit{GitLab}. Iako je ovaj udeo značajno manji, on ukazuje na prisustvo dela zajednice koji se fokusira na specifične \textit{DevOps} funkcionalnosti ili rad u okviru privatnih repozitorijuma koje GitLab tradicionalno podržava.
    \item \textbf{Zaključak o homogenosti uzorka:} Dobijeni rezultati pokazuju da je grupa korisnika \textit{IntelliJ} alata prilično ujednačena u pogledu korišćenja pratećih alata, sa izrazitom preferencijom ka globalnom standardu koji predstavlja \textit{GitHub}.
\end{itemize}


\subsection{Analiza uticaja radnog iskustva na promenu razvojnog alata}

U ovom delu istraživanja analizira se korelacija između dužine radnog staža ispitanika i njihove tendencije ka promeni primarnog razvojnog alata. Rezultati prikazani na slici \ref{fig:iskustvo_promena} pružaju uvid u to koliko su programeri u različitim fazama karijere skloni usvajanju novih tehnologija.

\begin{figure}[H]
    \centering
    \includegraphics[width=0.85\textwidth]{Slike/odnos_promene_razvojnog_alata_i_radnog_iskustva.png}
    \caption{Uticaj radnog iskustva na promenu razvojnog alata}
    \label{fig:iskustvo_promena}
\end{figure}

Na osnovu podataka prikazanih na slici \ref{fig:iskustvo_promena}, analizirana je spremnost ispitanika da promene svoj primarni razvojni alat u zavisnosti od godina radnog iskustva. Rezultati ukazuju na sledeće trendove:

\begin{itemize}
    \item \textbf{Iskusni programeri (10--15 godina):} Ova grupa pokazuje izuzetno visoku stopu promene alata. Od ukupno 6 ispitanika, čak 5 je prešlo na novo okruženje, što sugeriše da iskusniji kadar ne okleva da usvoji modernija rešenja ukoliko ona nude veću efikasnost u radu.
    \item \textbf{Grupa sa 6--9 godina iskustva:} Kod ovih ispitanika je takođe uočen visok stepen prilagođavanja, gde je 9 od 14 programera promenilo svoj primarni alat.
    \item \textbf{Srednji nivo iskustva (3--5 godina):} Ovo je najbrojnija grupa u uzorku, u kojoj je uočljiva najveća podeljenost u navikama. Od 22 ispitanika, 10 je promenilo alat, dok je 12 ostalo pri svom prvobitnom izboru.
    \item \textbf{Početnici (manje od 1 godine):} Iako je uzorak u ovoj kategoriji minimalan, zabeleženo je da se promena alata dešava čak i u najranijoj fazi karijere.
\end{itemize}

Ukupni podaci ukazuju na to da programeri sa više radnog staža (preko 6 godina) zapravo češće menjaju alate nego što se to možda intuitivno očekuje. Razlog za ovo verovatno leži u evoluciji projekata na kojima rade, kao i u pojavi naprednijih alata baziranih na novim tehnologijama poput veštačke inteligencije.

%TODO staviti tabelu
\subsection{Analiza važnosti faktora kod korisnika tekstualnih editora}

Pored kompleksnih integrisanih razvojnih okruženja (IDE), značajan deo ispitanika oslanja se na tekstualne editore. Na slici \ref{fig:tekst_editori} prikazano je kako korisnici različitih editora ocenjuju ključne faktore prilikom izbora svog primarnog alata.

\begin{figure}[H]
    \centering
    \includegraphics[width=1\textwidth]{Slike/vaznost_faktora_u_odnosu_na_tekstualne_editore.png}
    \caption{Važnost faktora u odnosu na najčešće korišćene tekstualne editore}
    \label{fig:tekst_editori}
\end{figure}

Na osnovu prikupljenih odgovora, za svaki od analiziranih tekstualnih editora izračunata je prosečna ocena zadovoljstva korisnika. Ova vrednost predstavlja aritmetičku sredinu šest ključnih kriterijuma (brzina, jednostavnost, univerzalnost, dodaci, vizuelni izgled i AI trendovi), a računa se po formuli:

\begin{equation}
\bar{X}_{E} = \frac{f_{brzina} + f_{jednost} + f_{univ} + f_{plugin} + f_{visual} + f_{AI}}{6}
\end{equation}

Analiza podataka sa slike \ref{fig:tekst_editori} ukazuje na sledeće prosečne ocene po alatima:

\begin{itemize}
    \item \textbf{Typora:} $\bar{X} \approx 4.83$ — Alat sa najvišim prosekom, gde su skoro svi faktori, uključujući i AI trendove, ocenjeni maksimalnim ocenama.
    \item \textbf{NeoVim i Vim:} $\bar{X} \approx 3.65$ — Ovi editori beleže ekstremno visoke ocene za brzinu i dostupnost dodataka, ali ukupni prosek vuku naniže nešto slabije ocene za vizuelni izgled i AI integracije.
    \item \textbf{Sublime Text:} $\bar{X} \approx 3.75$ — Karakteriše ga visok stepen ujednačenosti, sa posebnim naglaskom na brzinu i jednostavnost korišćenja.
    \item \textbf{Notepad++:} $\bar{X} \approx 3.73$ — Korisnici ovog alata najviše vrednuju univerzalnost i jednostavnost, dok su AI trendovi očekivano slabije ocenjeni.
    \item \textbf{Kate:} $\bar{X} \approx 3.63$ — Solidan prosek koji održavaju stabilne ocene za brzinu i univerzalnost formata.
    \item \textbf{Micro i TextEdit:} $\bar{X} \approx 3.50$ — Ovi editori se percipiraju kao jednostavna i brza rešenja, ali bez značajnog fokusa na proširivost ili moderan UX.
\end{itemize}

Ova analiza potvrđuje da korisnici tekstualnih editora prioritet daju performansama i brzini rada. Iako su modernija rešenja poput \textit{Typora-e} visoko ocenjena u svim kategorijama, tradicionalna zajednica (Vim, Notepad++) ostaje verna alatima koji nude stabilnost i efikasnost, čak i uz svesno žrtvovanje vizuelnog ugođaja ili najnovijih tehnoloških dodataka.

%TODO neka se ne grana po fakultetima nego globalno sammo analiza
\subsection{Uporedna analiza prioriteta: MATF, FON i RAF}

Poseban segment istraživanja posvećen je analizi prioriteta ispitanika u zavisnosti od fakulteta na kojem studiraju ili su diplomirali. Na slikama \ref{fig:matf_prioriteti}, \ref{fig:fon_prioriteti} i \ref{fig:raf_prioriteti} prikazani su rezultati za Matematički fakultet (MATF), Fakultet organizacionih nauka (FON) i Računarski fakultet (RAF).

\begin{figure}[H]
    \centering
    \includegraphics[width=0.85\textwidth]{Slike/matf_sta_mu_najbitnije.png}
    \caption{Prioriteti ispitanika sa Matematičkog fakulteta}
    \label{fig:matf_prioriteti}
\end{figure}

\begin{figure}[H]
    \centering
    \includegraphics[width=0.85\textwidth]{Slike/fon_sta_mu_najbitnije.png}
    \caption{Prioriteti ispitanika sa Fakulteta organizacionih nauka}
    \label{fig:fon_prioriteti}
\end{figure}

\begin{figure}[H]
    \centering
    \includegraphics[width=0.85\textwidth]{Slike/raf_sta_mu_najbitnije.png}
    \caption{Prioriteti ispitanika sa Računarskog fakulteta}
    \label{fig:raf_prioriteti}
\end{figure}

Komparativnom analizom prosečnih ocena važnosti $ \bar{X} $ za ove tri grupe, mogu se uočiti sledeći trendovi:

\begin{itemize}
    \item \textbf{Zajednički konsenzus o performansama:} Kod svih grupa, \textit{Brzina i odziv} je najviše rangiran faktor sa ocenama iznad $4.5$. Ovo potvrđuje univerzalnu potrebu programera za alatima koji ne usporavaju proces pisanja koda.
    \item \textbf{Razlike u pragmatičnosti (Jednostavnost vs. Univerzalnost):} 
    \begin{itemize}
        \item Ispitanici sa \textbf{MATF-a} (na bazi 32 ispitanika) visoko vrednuju \textit{Jednostavnost korišćenja} ($\approx 4.12$), stavljajući je ispred univerzalnosti alata.
        \item Ispitanici sa \textbf{RAF-a} daju podjednako visok značaj \textit{Univerzalnosti} i \textit{Brzini} (obe $\approx 4.67$), što može ukazivati na rad u različitim tehnologijama.
        \item Ispitanici sa \textbf{FON-a} najviše vrednuju \textit{Dostupnost plugin-ova} ($\approx 4.0$) odmah nakon brzine, što sugeriše sklonost ka prilagođavanju okruženja specifičnim potrebama.
    \end{itemize}
    \item \textbf{Odnos prema AI trendovima:} Uočen je zanimljiv trend najnižih ocena za faktor \textit{Praćenje AI trendova}. Dok je na MATF-u taj prosek oko $2.44$, na RAF-u pada na samo $1.67$, a na FON-u je nešto viši ($\approx 2.67$). Ovi podaci sugerišu da je akademska zajednica i dalje primarno fokusirana na stabilnost i jezgro funkcionalnosti alata u odnosu na trenutne AI inovacije.
    \item \textbf{Vizuelni izgled:} Najmanju važnost estetskom izgledu pridaju ispitanici sa FON-a ($\approx 3.33$), dok je na MATF-u i RAF-u taj prosek nešto viši ($\approx 3.67 - 3.75$).
\end{itemize}

Iako su uzorci za FON i RAF znatno manji u odnosu na MATF, poređenje ukazuje na specifične nijanse u profilisanju studenata: dok MATF teži balansu jednostavnosti i performansi, RAF naglašava univerzalnost, a FON fleksibilnost kroz ekstenzije.

\subsection{Uporedna analiza prioriteta: Studenti nasuprot zaposlenih}

Finalni segment analize fokusira se na razlike u prioritetima između studenata (22 ispitanika) i zaposlenih u IT sektoru (21 ispitanik). Poređenje rezultata prikazanih na slikama \ref{fig:studenti_bitno} i \ref{fig:zaposleni_bitno} omogućava uvid u evoluciju potreba programera tokom karijere.

\begin{figure}[H]
    \centering
    \includegraphics[width=0.85\textwidth]{Slike/studenti_sta_im_najbitnije.png}
    \caption{Faktori koji su najbitniji studentima}
    \label{fig:studenti_bitno}
\end{figure}

\begin{figure}[H]
    \centering
    \includegraphics[width=0.85\textwidth]{Slike/zaposleni_sta_mu_najbitnije.png}
    \caption{Faktori koji su najbitniji zaposlenima u IT sektoru}
    \label{fig:zaposleni_bitno}
\end{figure}

Na osnovu uporednog pregleda srednjih ocena važnosti $ \bar{X} $, uočavaju se sledeće ključne razlike:

\begin{itemize}
    \item \textbf{Neprikosnovenost performansi:} Obe grupe su \textit{Brzinu i odziv} ocenile gotovo identično visokom ocenom ($\approx 4.67$), što potvrđuje da je efikasnost alata apsolutni prioritet bez obzira na profesionalni status.
    \item \textbf{Jednostavnost vs. Profesionalni UX:} 
    \begin{itemize}
        \item \textbf{Studenti} daju znatno viši značaj \textit{Jednostavnosti korišćenja} ($\approx 4.14$), što je logično s obzirom na proces učenja i potrebu za brzim savladavanjem alata.
        \item \textbf{Zaposleni} pokazuju blagi pad važnosti jednostavnosti ($\approx 3.95$), ali zato daju veću važnost faktoru \textit{Vizuelni izgled / UX} ($\approx 3.90$ naspram studentskih $3.55$), što ukazuje na to da profesionalci koji provode više vremena u alatima više vrednuju ergonomiju i dizajn interfejsa.
    \end{itemize}
    \item \textbf{Ekstenzibilnost i univerzalnost:} Zaposleni pokazuju veću ujednačenost u vrednovanju univerzalnosti i dostupnosti plugin-ova, dok studenti daju blagu prednost univerzalnosti alata.
    \item \textbf{Odnos prema AI inovacijama:} Interesantno je da \textbf{zaposleni} daju veću ocenu \textit{Praćenju AI trendova} ($\approx 2.57$) u odnosu na \textbf{studente} ($\approx 2.32$). Ovo sugeriše da profesionalni sektor brže prepoznaje praktičnu vrednost AI asistenata u produkcionom kodu.
\end{itemize}

Zaključno, dok su osnovni zahtevi za brzinom isti, prelazak u profesionalnu sferu donosi veći fokus na korisničko iskustvo (UX) i AI integracije, dok važnost puke jednostavnosti korišćenja blago opada u korist naprednijih funkcionalnosti.

\FloatBarrier

\section{Diskusija}
U ovom radu su analizirani obrasci korišćenja razvojnih alata među studentima tehničkih fakulteta i profesionalcima u IT sektoru, sa posebnim fokusom na kriterijume izbora i spremnost na promenu alata. Dobijeni rezultati se u velikoj meri poklapaju sa literaturom, ali ukazuju i na određene specifičnosti konteksta.

\subsection{Najkorišćeniji razvojni alatii}
U ovom delu analizirani su najkorišćeniji razvojni alati među ispitanicima, a rezultati su upoređeni sa relevantnom literaturom, uključujući izvore \cite{top27} za IDE i tekstualne editore, \cite{git} za sisteme za verzionisanje, kao i \cite{github.gitlab} za platforme poput GitHub-a i GitLab-a.

\begin{itemize}
\item\textbf{IDE alati:} Među našim ispitanicima je najkorišćeniji alat \textit{Visual Studio/Visual Studio Code}.
Pored toga, razvojna okruženja iz \textit{JetBrains} porodice imaju izraženu zastupljenost u našem uzorku, gde \textit{IntelliJ IDEA} koristi 46,5\% ispitanika, \textit{PyCharm} 11,6\%, a \textit{CLion} 14\%. Ovi rezultati se u velikoj meri poklapaju sa literaturom.


Međutim, određena odstupanja u odnosu na literaturu su uočljiva.
\item \textbf{Tekstualni editori:} Najzastupljeniji tekstualni editori kod naših ispitanika su \textit{Notepad++} (32,6\%), \textit{Vim} (27,9\%) i \textit{Kate} (25,6\%), dok \textit{Sublime Text} koristi svega 9,3\% ispitanika. U literaturi je, međutim, \textit{Sublime Text} rangiran na drugom mestu najkorišćenijih razvojnih alata, što ukazuje na značajno odstupanje naših rezultata od opštih trendova. Ova razlika može se objasniti uticajem akademskog okruženja i ličnih navika korisnika, kao i relativno malim obimom uzorka.


\item \textbf{Sistemi za verzionisanje koda:} Kada je u pitanju verzionisanje koda, \textit{Git} se izdvaja kao ubedljivo najpopularniji sistem, kako prema literaturi, tako i u našem istraživanju, gde ga koristi čak 100\% ispitanika. Ovakav rezultat potvrđuje status Git-a kao industrijskog standarda u savremenom razvoju softvera, nezavisno od korišćenog razvojnog okruženja ili nivoa iskustva programera.
U našoj anketi se jasno vidi znatna dominacija GitHub-a nad GitLab-om kao platformama za upravljanje verzijama koda — GitHub koristi 88,4\% ispitanika, dok GitLab koristi 11,6\%. Ovaj rezultat ukazuje na snažnu preferencu prema GitHub-u i u našem uzorku, što se poklapa sa opštim trendom u IT industriji, gde GitHub generalno ima znatno veći udeo u korišćenju u odnosu na druge Git platforme.

\end{itemize}

%TODO dodati diskusiju o onim ocenama /6
\subsection{Faktori koij utiču na izbor alata}
Pored same zastupljenosti pojedinih alata, rezultati istraživanja pružaju uvid u obrasce ponašanja i navike programera prilikom izbora i promene razvojnog okruženja. Analiza faktora izbora pokazuje da su \textit{brzina i odziv} alata dosledno ocenjeni kao najvažniji kriterijum, nezavisno od pola, radnog statusa ili nivoa iskustva ispitanika. Ovakav rezultat potvrđuje da programeri primarno vrednuju efikasnost i stabilnost, budući da razvojni alati predstavljaju centralni deo svakodnevnog rada.
Takođe, analiza spremnosti na promenu razvojnog alata ukazuje da programeri sa većim radnim iskustvom češće menjaju svoje primarno okruženje. Ovo sugeriše da se sa razvojem karijere smanjuje vezanost za konkretan alat, dok raste fokus na prilagodljivost i efikasnost u različitim tehnološkim kontekstima. Nasuprot tome, manje iskusni programeri češće ostaju pri jednom alatu, što može biti posledica formiranja početnih navika i potrebe za stabilnim okruženjem tokom učenja.

Analiza faktora koji utiču na izbor razvojnog alata pokazuje da dugogodišnje navike imaju značajan uticaj. Većina ispitanika je ocenila ovaj faktor visoko — čak 62,8\% je označilo ocenu 4, a 23,3\% ocenu 5, dok samo 4,6\% ispitanika smatra da navike uopšte ne utiču ili imaju mali uticaj. Ovo ukazuje da programeri često ostaju pri alatima koje poznaju i sa kojima imaju iskustvo, čak i kada postoje alternative sa sličnim ili boljim funkcionalnostima.

U celini, rezultati istraživanja ukazuju da izbor razvojnog alata nije zasnovan isključivo na tehničkim karakteristikama, već predstavlja kombinaciju profesionalnih potreba, stečenih navika i konteksta u kojem programer radi ili se obrazuje.

\subsection{Ograničenja istraživanja}
Iako dobijeni rezultati pružaju uvid u aktuelne trendove korišćenja razvojnih alata i preference korisnika, istraživanje ima nekoliko ograničenja koja treba uzeti u obzir pri tumačenju rezultata.  

Prvo, uzorak od 43 ispitanika nije dovoljno obiman da bi se izvukli generalni zaključci za širu populaciju programera. Za značajnije statističke procene i pouzdano određivanje trendova, bio bi potreban znatno veći i raznovrsniji uzorak.  

Drugo, sastav uzorka je specifičan — većina ispitanika su studenti Matematičkog fakulteta, što može povući pristrasnost prema određenim alatima, budući da se na fakultetu koriste uglavnom određeni razvojni alati i okruženja.  

Treće, rezultati su zasnovani na samoproceni ispitanika, što može biti pod uticajem subjektivnih stavova, navika ili trenutnog iskustva. Navike korišćenja alata, prethodno iskustvo i lične preferencije mogu značajno uticati na odgovore, što ograničava mogućnost da se dobijeni podaci tretiraju kao apsolutno reprezentativni.  

Konačno, anketa nije obuhvatila sve moguće faktore koji utiču na izbor alata, kao što su specifični zahtevi projekata, industrijski standardi u kompanijama ispitanika ili regionalne razlike, što znači da određene varijable koje mogu biti značajne nisu uzete u obzir.  

Uzimajući sve ovo u obzir, rezultati treba interpretirati sa oprezom i kao okvirne smernice, a ne kao definitivan prikaz navika i preferencija svih programera.

\section{Zaključak}

Cilj ovog seminarskog rada bio je da se ispita koji razvojni alati se najčešće koriste među studentima i profesionalcima u IT sektoru, kao i koji faktori imaju najveći uticaj na njihov izbor. Na osnovu sprovedene ankete i analize dobijenih rezultata, može se zaključiti da su \textit{Visual Studio} i \textit{Visual Studio Code} dominantni alati u praksi, što je u skladu sa savremenim industrijskim trendovima i relevantnom literaturom.

Istraživanje je pokazalo da su performanse alata, odnosno brzina i odziv, ključni kriterijumi pri izboru razvojnog okruženja, dok faktori poput vizuelnog izgleda i praćenja AI trendova imaju manji, ali ne zanemarljiv značaj. Uočene razlike između studenata i zaposlenih ukazuju na promenu prioriteta tokom profesionalnog razvoja.

Rezultati takođe potvrđuju dominaciju sistema \textit{Git} kao standarda za verzionisanje koda, kao i znatno veću popularnost platforme \textit{GitHub} u odnosu na \textit{GitLab}. Ovakvi nalazi dodatno naglašavaju homogenost savremenih razvojnih praksi i oslanjanje na globalno prihvaćene alate i platforme.

Iako istraživanje ima određena ograničenja, pre svega u pogledu veličine i strukture uzorka, dobijeni podaci pružaju relevantan uvid u aktuelne navike i preferencije programera. Ovaj rad može poslužiti kao osnova za buduća, obimnija istraživanja koja bi uključila veći broj ispitanika, širi spektar tehnologija i dublju analizu uticaja novih trendova, poput veštačke inteligencije, na izbor razvojnih alata.


\addcontentsline{toc}{section}{Literatura}
\renewcommand{\refname}{Literatura}
\begin{thebibliography}{10}
\bibliographystyle{unsrt}
\bibitem{top27} Top 27 Software Development Tools and Platforms (2025 List) at: \\ \url{https://spacelift.io/blog/software-development-tools}
\bibitem{git} Version Control Systems at: \\ \url{https://www.geeksforgeeks.org/git/version-control-systems/}
\bibitem{github.gitlab} GitLab vs GitHub at: \\ \url{https://evbn.org/gitlab-vs-github-explore-their-major-differences-and-similarities-1678449077/}
\end{thebibliography}
\end{document}