\documentclass[xcolor=svgnames,t]{beamer}

\usepackage[T1]{fontenc}
\usepackage[utf8]{inputenc}
\usepackage[serbian]{babel}

\usepackage{amsmath}
\usepackage[makeroom]{cancel}
\usepackage{booktabs,comment}
\usepackage{csquotes}
\usepackage{graphicx}
\usepackage{tikz}
\usepackage[absolute,overlay]{textpos}
\usepackage[font=footnotesize]{caption}

\usetheme{Madrid}
\useoutertheme{infolines}
\setbeamertemplate{page number in head/foot}{}
\setbeamersize{text margin left=1cm, text margin right=1cm}

\definecolor{matfDarkRed}{RGB}{128, 0, 32}   
\definecolor{matfRed}{RGB}{178, 34, 34}      
\definecolor{matfBlack}{RGB}{0, 0, 0}       

\usecolortheme[named=matfDarkRed]{structure}

\setbeamercolor{normal text}{fg=matfBlack,bg=white}

\setbeamercolor{frametitle}{fg=white,bg=matfDarkRed}

\setbeamercolor{title in head/foot}{bg=matfRed,fg=white}
\setbeamercolor{author in head/foot}{bg=matfBlack,fg=white}
\setbeamercolor{date in head/foot}{bg=matfBlack,fg=white}

\setbeamercolor{block title}{bg=matfDarkRed,fg=white}
\setbeamercolor{block body}{bg=white,fg=matfBlack}

\addtobeamertemplate{navigation symbols}{}{%
  \usebeamerfont{footline}%
  \usebeamercolor[fg]{footline}%
  \hspace{1em}%
  \insertframenumber/\inserttotalframenumber
}

\newcommand{\matflogo}{Slike/grb-matf.png}

\title[Najkorišćeniji razvojni alati]{Najkorišćeniji razvojni alati}
\subtitle{Seminarski rad u okviru kursa Metodologija stručnog i naučnog rada}
\author[Matematički fakultet]{
Staša Đorđević, Lazar Savić, Đurđa Milošević, Bogdan Tomić
}
\institute[]{Matematički fakultet \\ Univerzitet u Beogradu}
\date{\today}

\titlegraphic{\includegraphics[height=2.0cm]{\matflogo}}

\begin{document}
{
\begin{frame}
  \maketitle
\end{frame}
}
\begin{frame}
  \frametitle{Sadržaj}
  \tableofcontents
\end{frame}

\section{Uvod}
\begin{frame}{Uvod}
\vspace{0.5cm}

\begin{itemize}
  \item Razvojni alati imaju ključnu ulogu u produktivnosti i efikasnosti programera
  \item Neadekvatan izbor alata može usporiti rad i otežati razvoj softvera
  \item Veliki broj alata → izbor često zavisi od iskustva i navike
  \item Dosadašnja istraživanja izdvajaju VS Code, Sublime Text i Git kao najkorišćenije alate
  \item Cilj istraživanja je da se identifikuju najčešće korišćeni alati i faktori koji utiču na njihov izbor
\end{itemize}
\vspace*{0.5cm}
\end{frame}

\section{O razvojnim alatima}
\begin{frame}{O razvojnim alatima}
\vspace{0.5cm}

\begin{itemize}
  \item U istraživanju su razmatrane tri osnovne kategorije razvojnih alata:
  \begin{itemize}
    \item \textbf{Integrisana razvojna okruženja (IDE)} - objedinjuju pisanje, testiranje i debagovanje koda
    \item \textbf{Tekstualni editori koda} - namenjeni pisanju i uređivanju izvornog koda, sa fokusom na jednostavnost i fleksibilnost
    \item \textbf{Sistemi za verzionisanje} - zaduženi za čuvanje i kontrolisanje izmena programskog koda
  \end{itemize}
\end{itemize}

\vspace{0.8cm}
\begin{columns}[c,onlytextwidth]
  \column{0.33\textwidth}
    \centering
    \includegraphics[height=1.5cm]{Slike/vsc.png}

  \column{0.33\textwidth}
    \centering
    \includegraphics[height=1.5cm]{Slike/sublime-text.png}

  \column{0.33\textwidth}
    \centering
    \includegraphics[height=1.1cm]{Slike/git.png}
\end{columns}

\vspace*{0.5cm}
\end{frame}

\section{Metodologija}
\begin{frame}{Metodologija}
\vspace{0.5cm}

\begin{itemize}
  \item Anonimna onlajn anketa
  \item Dominantno kvantitativni istraživački pristup
  \item \textbf{Struktura uzorka}
  \begin{itemize}
    \setlength{\itemsep}{0.15em}
    \item Obim: \textbf{43 ispitanika}
    \item Studenti tehničkih fakulteta i iskusniji programeri
    \item 27 muškog pola i 16 ženskog
    \item Većinski sa Matematičkog fakulteta (79,1\%)
    \item Najčešće 3–5 godina iskustva (51,2\%)
  \end{itemize}
  \item \textbf{Struktura ankete}
  \begin{itemize}
    \item 15 pitanja: 12 zatvorenog i 3 otvorenog tipa
  	\item Demografska i profesionalna pitanja (pol, iskustvo, fakultet)
  	\item Pitanja o korišćenim razvojnim alatima i faktorima izbora
  \end{itemize}
\end{itemize}

\vspace*{0.5cm}
\end{frame}

\begin{frame}{Metodologija}

\vfill

\begin{columns}[c,onlytextwidth]
  \column{0.5\textwidth}
    \centering
    \includegraphics[width=0.85\linewidth]{Slike/fakulteti.png}

  \column{0.5\textwidth}
    \centering
    \includegraphics[width=0.85\linewidth]{Slike/duzina iskustva.png}
\end{columns}

\vfill

\vspace*{0.5cm}
\end{frame}

\section{Rezultati}

\begin{frame}{Rezultati}
\vspace{0.5cm}

\begin{itemize}
  \item U okviru istraživanja razmatrani su sledeći aspekti:
  \begin{itemize}
    \item najkorišćeniji razvojni alati među ispitanicima
    \item faktori koji utiču na izbor alata
    \item sklonost ka promeni alata
    \item analiza otvorenih odgovora ispitanika
  \end{itemize}
  \item Dobijeni rezultati su upoređeni sa nalazima iz relevantne literature
  \item Istraživanje pruža uvid u navike i preferencije korisnika razvojnih alata
\end{itemize}

\vspace*{0.5cm}
\end{frame}


\subsection{Najkorišćeniji razvojni alati}
\begin{frame}{Najkorišćeniji razvojni alati}
\vspace{0.5cm}

\begin{itemize}
  \item IDE: \textbf{Visual Studio / Visual Studio Code}
  \begin{itemize}
    \item dominantni među ispitanicima
    \item u skladu sa dosadašnjim istraživanjima
  \end{itemize}

  \item Tekstualni editori: \textbf{Notepad++}
  \begin{itemize}
    \item odstupanje sa literaturom koja izdvaja \textbf{Sublime Text} (u uzorku zastupljen samo 9,3\%)
  \end{itemize}

  \item Sistemi za verzionisanje: \textbf{Git}
  \begin{itemize}
    \item platformu \textbf{GitHub} koristi 88,4\% ispitanika
    \item u skladu sa opštim trendovima
  \end{itemize}
\end{itemize}

\vspace*{0.5cm}
\end{frame}
\subsection{Faktori izbora alata}
\begin{frame}{Faktori izbora alata}
\vspace{0.5cm}

\begin{itemize}
  \item \textbf{Brzina} i \textbf{odziv okruženja} imaju najvišu prosečnu ocenu (4.74/5)
  \item Rezultati ukazuju da ispitanici preferiraju alate koji omogućavaju brži i efikasniji svakodnevni rad (ušteda vremena)
  \item \textbf{Dugogodišnje navike} imaju snažan uticaj na izbor razvojnog alata
\end{itemize}

\vspace{0.2cm}

\begin{center}
  \includegraphics[width=0.8\textwidth]{Slike/top_5_konkretnih_alataa.png}
\end{center}

\vspace*{0.5cm}
\end{frame}

\begin{frame}{Faktori izbora alata: studenti vs zaposleni}
\vspace{0.5cm}

\begin{itemize}
  \item Uočene su razlike u prioritetima između studenata i zaposlenih programera
  \item Studenti više vrednuju jednostavnost korišćenja
  \item Zaposleni više vrednuju UX dizajn i praćenje AI trendova
\end{itemize}

\vspace{0.4cm}

\begin{center}
  \includegraphics[width=0.7\textwidth]{Slike/oredjenje_prioriteta_studenti_zaposleni.png}
\end{center}

\vspace*{0.5cm}
\end{frame}

\subsection{Sklonost ka promeni alata}
\begin{frame}{Sklonost ka promeni alata}
\vspace{0.5cm}

\begin{itemize}
  \item Veće radno iskustvo (posebno \textbf{6+ godina}) povećava sklonost ka promeni razvojnog alata
  \item Visok značaj navike  $\neq$  otpor prema promeni
  \item Promena alata se javlja kada novi alat donosi jasne funkcionalne prednosti
\end{itemize}

\vspace{0.1cm}

\begin{center}
  \includegraphics[width=0.7\textwidth]{Slike/odnos_promene_alata_i_radnog_iskustva.jpeg}
\end{center}
\vspace*{0.5cm}
\end{frame}
\subsection{Analiza otvorenih odgovora}
\begin{frame}{Analiza otvorenih odgovora}
\vspace{0.5cm}

\begin{itemize}
  \item Otvoreni odgovori dopunjuju kvantitativne rezultate istraživanja
  \item \textbf{Razlozi izbora alata:}
  \begin{itemize}
    \item balans jednostavnosti i funkcionalnosti (VS Code)
    \item snažna podrška za konkretne jezike i okvire (JetBrains alati)
    \item zahtevi fakulteta ili radnog okruženja
  \end{itemize}
  \item \textbf{Uočeni problemi:}
  \begin{itemize}
    \item visoka potrošnja resursa kod kompleksnih IDE
    \item ograničene napredne funkcije kod lakših editora
  \end{itemize}
  \item Izbor alata zasniva se na kompromisu između performansi, funkcionalnosti i navike
\end{itemize}

\vspace*{0.5cm}
\end{frame}

\section{Zaključak}

\begin{frame}{Zaključak}
\vspace{0.5cm}

\begin{itemize}
  \item Najkorišćeniji IDE: \textbf{Visual Studio / Visual Studio Code}, a tekstualni editor \textbf{Notepad++}
  \item Ubedljiva dominacija alata \textbf{Git} u sistemima za verzionisanje
  \item \textbf{Brzina i odziv okruženja} su ključni kriterijumi pri izboru alata
  \item Navika ima značajan uticaj na izbor alata i kod studenata i kod zaposlenih programera
  \item Iskusniji programeri češće menjaju razvojne alate od početnika
\end{itemize}

\vspace*{0.5cm}
\end{frame}

\section{Literatura}
\begin{frame}[allowframebreaks]{Literatura}
\footnotesize
\nocite{*}
\bibliographystyle{apalike}
\bibliography{seminarski}

\end{frame}

\end{document}
